\documentclass[a4paper,12pt]{article}
\usepackage{amsmath,amsfonts,amsthm,amssymb, mathtools,steinmetz, gensymb, siunitx}	% LOADS USEFUL MATH STUFF
\usepackage{xcolor,graphicx}
\usepackage[a4paper]{geometry} 				% ADJUSTS PAGE
\usepackage{setspace}
\usepackage{physics}
\usepackage{caption}
\usepackage{tikz}
\usepackage{pgf,tikz,pgfplots}
\usepackage{mathrsfs}
\usepackage{amsbsy}
\usepackage{fancyhdr}
\usepackage{float}
\usepackage{array}
\usepackage{booktabs}
\usepackage{newpxtext}
\usepackage{unicode-math}
\usepackage{braket}
\usepackage{tensor}
\setmathfont{Libertinus Math}

\usetikzlibrary{decorations.pathreplacing,decorations.markings}
\usepgfplotslibrary{fillbetween}

\newgeometry{left=1cm,top = 2.5cm, bottom = 1.75cm, right = 1cm}

\newcommand{\defeq}{:=}
\newcommand\block[1]{\hspace*{#1}}
\newcommand{\rpm}{\sbox0{$1$}\sbox2{$\scriptstyle\pm$}
	  \raise\dimexpr(\ht0-\ht2)/2\relax\box2 }
\newcommand{\af}{\pmb{\hat a_1}}
\newcommand{\as}{\pmb{\hat a_2}}
\newcommand{\at}{\pmb{\hat a_3}}
\newcommand\uv[1]{\pmb{\hat {#1}}}
\newcommand\vect[1]{\pmb{{#1}}}
\newcommand\dprod{\pmb{\cdot}}

\usepackage{tabstackengine}
\setstackEOL{;}% row separator
\setstackTAB{,}% column separator
\setstacktabbedgap{1ex}% inter-column gap
\setstackgap{L}{1.0\normalbaselineskip}% inter-row baselineskip
\newcommand\rbm[1]{\left(\Matrixstack{#1}\right)}
	  
\pgfplotsset{compat=newest}
\newlength{\QNo}
\settowidth{\QNo}{2.}

\newlength{\QLetter}
\settowidth{\QLetter}{(a)}

\pagestyle{fancy}
\rhead{Quantum mechanics Problem Set}
\lhead{J. L. Gouws}


\begin{document}
\fontencoding{T1}
\fontfamily{ppl}\selectfont
{\Large \textbf{Quantum mechanics Assignment 2}} \hfill {\Large \textbf{J L Gouws}}\\
\block{1.0cm} {\large \textbf{\today}} \hfill {\large \textbf{26634554}}\\
\thispagestyle{empty}
\fontencoding{T1}

1.
\begin{minipage}[t]{0.90\textwidth}
  a)
  \begin{minipage}[t]{\textwidth}
    For this, first note that:
    \begin{align*}
      U^\dagger U &= (A - \langle A \rangle) (A - \langle A \rangle)\\
                  &= A^2 -2 A\langle A\rangle + \langle A \rangle^2\\
      \Rightarrow & \bra{\psi} U^\dagger U \ket{\psi}\\
                  & = \braket{\psi|A^2|\psi} - 2\langle A \rangle \braket{\psi|A\psi} + \langle A \rangle^2 \braket{\psi|\psi}\\
                  & = \braket{\psi|A^2|\psi} - 2\langle A \rangle \langle A \rangle + \langle A \rangle^2 \\
                  & = \braket{\psi|A^2|\psi} - \langle A \rangle^2\\
                  & = \langle (\Delta A)^2 \rangle 
    \end{align*}
    Assuming that $\ket{\psi}$ is normalized. Similarly(swaping symbols) $ \braket{\psi|V^\dagger V|\psi} = \langle (\Delta B )^2 \rangle$. Also:
    \begin{align*}
     \braket{\psi | U^\dagger V|\psi} &= \braket{\psi |(A - \langle A \rangle) (B - \langle B \rangle)|\psi}\\
                  &= \braket{\psi|A B|\psi} - \langle A \rangle \braket{\psi|B| \psi}- \langle B \rangle \braket{\psi|A|\psi} + \langle A \rangle \langle B \rangle \\
                  &= \braket{\psi|A B|\psi} - \langle A \rangle \langle B \rangle
    \end{align*}
    Similarly $\braket{\psi | V^\dagger U|\psi} = \braket{\psi|B A|\psi} - \langle A \rangle \langle B \rangle $. Now we have:
    \begin{align*}
      \braket{\phi|\phi} &= \left(\bra{\psi} U^\dagger - i\lambda\bra{\psi}V^\dagger \right) \left(U\ket{\psi} + i\lambda V\ket{\psi}\right)\\
                         &= \bra{\psi} U^\dagger \left(U\ket{\psi} + i\lambda V\ket{\psi}\right)- i\lambda \bra{\psi}V^\dagger \left(U\ket{\psi} + i\lambda V \ket{\psi}\right)\\
                         &=  \bra{\psi} U^\dagger U\ket{\psi} + i\lambda \bra{\psi} U^\dagger V\ket{\psi}- i\lambda\bra{\psi}V^\dagger U\ket{\psi} + \lambda^2\bra{\psi}V^\dagger  V \ket{\psi}\\
                         &=  \langle (\Delta A)^2 \rangle + i\lambda \braket{\psi | AB |\psi} + i\lambda \langle A \rangle \langle B \rangle - i\lambda\braket {\psi| B A | \psi} - i\lambda \langle A \rangle \langle B \rangle+ \lambda^2\langle (\Delta B)^2 \rangle\\
                         &=  \langle (\Delta A)^2 \rangle + i\lambda \braket{\psi | AB - BA|\psi} + \lambda^2\langle (\Delta B)^2 \rangle\\
                         &=  \langle (\Delta A)^2 \rangle + i\lambda \langle [A,B]\rangle + \lambda^2\langle (\Delta B)^2 \rangle\\
    \end{align*}
  \end{minipage}

  b)
  \begin{minipage}[t]{\textwidth}
    $\braket{\phi|\phi}$ is minimized when $\frac{\partial}{\partial \lambda} \braket{\phi|\phi} = 0$:
    \begin{align*}
                  \frac{\partial}{\partial \lambda} \braket{\phi|\phi} = 0
      \Rightarrow & \frac{\partial}{\partial \lambda} \left[\langle (\Delta A)^2 \rangle + i\lambda \langle [AB]\rangle + \lambda^2\langle (\Delta B)^2 \rangle\right] = 0\\
      \Rightarrow & i\langle [A,B]\rangle + 2 \lambda \langle (\Delta B)^2 \rangle = 0\\
      \Rightarrow &  \lambda = - \frac{i\langle [A,B]\rangle }{2 \langle (\Delta B)^2 \rangle }
    \end{align*}
    In order for $\lambda$ to be real, $\langle [A,B]\rangle$ must be purely imaginary.
    This is true because:
    \begin{equation*}
      \left(\bra{} [A, B] \ket{}\right)^* = \bra{} [A, B]^\dagger \ket{} = \bra{} BA - AB \ket{} = -\bra{} AB - BA \ket{} = -\bra{} [A, B] \ket{}
    \end{equation*}
    Where I used unlabeled vectors to show that the above holds true for any vectors.
    Thus we can write:
    \begin{align*}
        & \langle[A, B] \rangle = \pm i |\langle[A, B] \rangle|\\
      \Rightarrow & \lambda = \pm \frac{|\langle[A, B] \rangle|}{2\langle (\Delta B)^2 \rangle }
    \end{align*}
  \end{minipage}
\end{minipage}

$\phantom{1.}$
\begin{minipage}[t]{0.90\textwidth}
  $\phantom{b)}$
  \begin{minipage}[t]{\textwidth}
    Now the equation for $\braket{\phi | \phi}$ becomes:
    \begin{equation*}
      \braket{\phi|\phi} =  \langle (\Delta A)^2 \rangle \mp \lambda |\langle [A,B]\rangle| + \lambda^2\langle (\Delta B)^2 \rangle\\
    \end{equation*}
    Since $\braket{\phi|\phi} \geq 0$:
    \begin{align*}
      & 0 \leq \langle (\Delta A)^2 \rangle - \frac{|\langle [A,B]\rangle |}{2\langle (\Delta B)^2 \rangle} |\langle [A,B]\rangle | + \frac{|\langle [A, B]\rangle|^2}{4\langle (\Delta B)^2 \rangle^2}\langle (\Delta B)^2 \rangle\\
        \Rightarrow & 0 \leq \langle (\Delta A)^2 \rangle - \frac{|\langle [A, B]\rangle|^2}{4\langle (\Delta B)^2 \rangle}\\
        \Rightarrow & \frac{1}{4} |\langle [A, B]\rangle|^2 \leq \langle (\Delta A)^2 \rangle \langle (\Delta B)^2 \rangle
    \end{align*}
    Which one can write in the form $\frac{1}{4} \langle i [A,B]\rangle^2 \leq \langle (\Delta A)^2 \rangle \langle (\Delta B)^2 \rangle$, but I think the meaning of the statement clearer in the prior form.\\
  \end{minipage} 

  c)
  \begin{minipage}[t]{\textwidth}
    This statement says that if two observables do not commute(not all eigenstates are simultaneous eigenstates), then there is a lower bound on the product of the deviations of the observables from their mean value.
    This states that the more precisely you know the value of one observable the less precisely you know the value of another non-commuting observable.

    An example is momentum and position with $[\hat x_i, \hat p_j] = i \hbar \delta_{ij}$, which we can substitute in the generalized uncertainty principle to obtain:
    \begin{equation*}
      \frac{1}{4} \hbar^2 \leq \langle (\Delta x_i)^2 \rangle \langle (\Delta p_i)^2 \rangle \Rightarrow \frac{\hbar}{2} \leq \Delta x_i \Delta p_i
    \end{equation*}
    Which is the ususal Heisenberg uncertainty principle.
    The more precisely one knows the position of the particle the less precisely one can know the momentum of the particle.
  \end{minipage}
\end{minipage}

2.
\begin{minipage}[t]{0.9\textwidth}
  a).
  \begin{minipage}[t]{\textwidth}
    Simply expanding:
    \begin{equation*}
      [A, BC] = ABC - BCA = ABC - BAC + BAC - BCA = (AB - BA)C + B(AC - CA) = [A,B]C - B[A, C]
    \end{equation*}\\
  \end{minipage}

  b).
  \begin{minipage}[t]{\textwidth}
    First I check if it holds for $n = 1$:
    \begin{equation*}
      [A, B] =  cI = 1cB^0
    \end{equation*}
    As required.
    Now assume that it holds for some $k \geq 1$:
    \begin{align*}
      [A, B^{k +1}] &= [A, B^{k} B]\\
                    &= [A, B^{k} B]\\
                    &= [A, B^{k}]B + B^k[A, B]\\
                    &= k c B^{k - 1}B + B^k c I\\
                    &= k c B^{k} + c B^k I\\
                    &= (k + 1) c B^{k}
    \end{align*}
    As required.\\
  \end{minipage}

  c).
  \begin{minipage}[t]{\textwidth}
    Using the linearity of the commutator:
    \begin{align*}
      [A, f(B)] &= [A, \sum_{j = 0}^{\infty}\frac{f^{(j)}(0)}{j!} B^j]\\
                &= \sum_{j = 0}^{\infty}\frac{f^{(j)}(0)}{j!} [A, B^j]\\
                &= [A, I] + \sum_{j = 1}^{\infty}\frac{f^{(j)}(0)}{j!} [A, B^j]\\
                &= \sum_{j = 1}^{\infty}\frac{f^{(j)}(0)}{j!} [A, B^j]\\
                &= \sum_{j = 1}^{\infty}\frac{f^{(j)}(0)}{j!} j c B^{j - 1}\\
                &= c \sum_{j = 1}^{\infty}\frac{f^{(j)}(0)}{(j-1)!} B^{j - 1}\\
                &= c \sum_{k = 0}^{\infty}\frac{f^{(k + 1)}(0)}{k!} B^{k}\\
                &= c \sum_{k = 0}^{\infty}\frac{[f']^{(k)}(0)}{k!} B^{k}\\
                &= c f'(B)
    \end{align*}
  \end{minipage}
\end{minipage}

2.
\begin{minipage}[t]{0.9\textwidth}
  d).
  \begin{minipage}[t]{\textwidth}
    For this question I use the notation:
    \begin{equation*}
      [A, B]_n \defeq [\cdots [A, \underbrace{B ], B, ],\cdots, B}_\text{$n$ times}]
    \end{equation*}
    To avoid cluttered equations.
    First I show, by induction:
    \begin{equation*}
      \frac{d^n}{dx^n}\left[e^{-x A} B e^{xA} \right] = e^{-x A} [B, A]_n e^{xA}
    \end{equation*}
    For $n = 1$
    \begin{align*}
      \frac{d}{dx}\left[e^{-x A} B e^{xA} \right] &= \frac{d}{dx}\left[e^{-x A} \right]B e^{xA} + e^{-x A} B \frac{d}{dx}\left[e^{xA}\right]\\
                                                  &= e^{-x A}(- A) B e^{xA} + e^{-x A} B e^{xA}A\\
                                                  &= e^{-x A} (- A B) e^{xA} + e^{-x A} B A e^{xA}\\
                                                  &= e^{-x A} (B A - A B) e^{xA}\\
                                                  &= e^{-x A} [B,A] e^{xA}
    \end{align*}
    Now assume it holds for some $k \geq 1$:
    \begin{align*}
    \frac{d^{k + 1}}{dx^{k + 1}}\left[e^{-x A} B e^{xA} \right] &= \frac{d}{dx}\left[e^{-x A} [B, A]_ke^{xA}\right]\\
                                                                  &= \frac{d}{dx}\left[e^{-x A}\right] [B, A]_k e^{xA}+ e^{-x A} [B, A]_k\frac{d}{dx}\left[e^{xA}\right]\\
                                                                  &= e^{-x A} (-A) [B, A]_k e^{xA}+ e^{-x A} [B, A]_k e^{xA}A\\
                                                                  &= e^{-x A} [B, A]_k Ae^{xA} + e^{-x A} (-A [B, A]_k) e^{xA}\\
                                                                  &= e^{-x A} ([B, A]_k A-A [B, A]_k )e^{xA}\\
                                                                  &= e^{-x A} [[B, A]_{k}, A] e^{xA}\\
                                                                  &= e^{-x A} [B, A]_{k + 1} e^{xA}
    \end{align*}
    As required.
    It now follows that:
    \begin{align*}
      e^{-x A} B e^{xA} &= \sum_{i = 0}^{\infty}\frac{\frac{d^{i}}{dx^{i}}\left[e^{-x A} B e^{xA} \right]_{x = 0}}{i!} x\\
                        &= \sum_{i = 0}^{\infty}\frac{e^{0} [B,A]_i e^{0} }{i!} x\\
                        &= \sum_{i = 0}^{\infty}\frac{[B,A]_i}{i!} x\\
                        &\Rightarrow e^{-A} B e^{A} = \sum_{i = 0}^{\infty}\frac{1}{i!}[B,A]_i
    \end{align*}
    As required.
  \end{minipage}
\end{minipage}

\end{document}
