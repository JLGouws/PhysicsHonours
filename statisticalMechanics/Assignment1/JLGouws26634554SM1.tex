\documentclass[a4paper,12pt]{article}

\usepackage{amsmath,amsfonts,amsthm,amssymb, mathtools,steinmetz, gensymb, siunitx}	% LOADS USEFUL MATH STUFF
\usepackage{xcolor,graphicx}
\usepackage[a4paper]{geometry} 				% ADJUSTS PAGE
\usepackage{setspace}
\usepackage{physics}
\usepackage{caption}
\usepackage{tikz}
\usepackage{pgf,tikz,pgfplots}
\usepackage{mathrsfs}
\usepackage{amsbsy}
\usepackage{fancyhdr}
\usepackage{float}
\usepackage{array}
\usepackage{booktabs}
\usepackage{newpxtext}
\usepackage{unicode-math}
\usepackage{braket}
\usepackage{tensor}
\usepackage{tabstackengine}
\setmathfont{Libertinus Math}
\usepackage[frak=euler]{mathalfa}

\usetikzlibrary{decorations.pathreplacing,decorations.markings}
\usepgfplotslibrary{fillbetween}

\newgeometry{left=1cm,top = 2.5cm, bottom = 1.75cm, right = 1cm}

\newcommand{\defeq}{:=}
\newcommand\block[1]{\hspace*{#1}}
\newcommand{\rpm}{\sbox0{$1$}\sbox2{$\scriptstyle\pm$}
	  \raise\dimexpr(\ht0-\ht2)/2\relax\box2 }
\newcommand{\af}{\pmb{\hat a_1}}
\newcommand{\as}{\pmb{\hat a_2}}
\newcommand{\at}{\pmb{\hat a_3}}
\newcommand\uv[1]{\pmb{\hat {#1}}}
\newcommand\vect[1]{\pmb{{#1}}}
\newcommand\dprod{\pmb{\cdot}}

\setstackEOL{;}% row separator
\setstackTAB{,}% column separator
\setstacktabbedgap{1ex}% inter-column gap
\setstackgap{L}{1.0\normalbaselineskip}% inter-row baselineskip
\newcommand\rbm[1]{\left(\Matrixstack{#1}\right)}
	  
\pgfplotsset{compat=newest}
\newlength{\QNo}
\settowidth{\QNo}{2.}

\newlength{\QLetter}
\settowidth{\QLetter}{(a)}

\pagestyle{fancy}
\rhead{Statistical Physics Problem Set}
\lhead{J. L. Gouws}


\begin{document}
\fontencoding{T1}
\fontfamily{ppl}\selectfont
{\Large \textbf{Statistical Physics Tutorial 1}} \hfill {\Large \textbf{J L Gouws}}\\
\block{1.0cm} {\large \textbf{\today}} \hfill {\large \textbf{26634554}}\\
\thispagestyle{empty}
\fontencoding{T1}

1.
\begin{minipage}[t]{0.9\textwidth}
  We have:
  \begin{align*}
    z \frac{ \partial \ln \mathfrak{Z}}{\partial z} = z\left. \frac{\partial \mathfrak{Z}}{\partial z} \right/ \mathfrak{Z} = \frac{z}{\mathfrak{Z}} \sum_{N} N z^{N - 1} Z_N = \sum_{N} N \frac{z^{N} Z_N}{\mathfrak{Z}}= \sum_{N} N \frac{e^{\beta \mu N} \sum_re^{-\beta E_r}}{\mathfrak{Z}} = \sum_{N} N P_N = \langle N \rangle\\
  \end{align*}
\end{minipage}

2.
\begin{minipage}[t]{0.9\textwidth}
  The cannonical partition function for an ideal gas is:
  \begin{align*}
    Z_N &= \frac{1}{h^3N!} \int d^{3N}p_{\ } d^{3N}qe^{-\beta H}\\
        &= \frac{1}{\lambda^{3N}N!} \int d^{3N}q \exp\left\{-\beta \sum\limits_{i < j} u_{ij}\right\}\\
        &= \frac{1}{\lambda^{3N}N!} \int d^{3N}q e^{-\beta \times 0}\\
        &= \frac{1}{\lambda^{3N}N!} \int d^{3N}q \\
        &= \frac{V^N}{\lambda^{3N}N!}
  \end{align*}  
  Since there is no interaction between ideal gas molecules.
  We can use this to evaluate the grand partion function for an ensemble of ideal gas samples:
  \begin{align*}
    \mathfrak{Z} &= \sum_{N = 0}^\infty z^N \frac{V^N}{\lambda^{3N}N!}\\
                 &= \sum_{N = 0}^\infty \frac{(zV/\lambda^3)^N}{N!}\\
                 &= e^{zV/\lambda^3}\\
  \end{align*}
  We also have that:
  \begin{equation*}
    P = \frac{k_BT}{V} \ln \mathfrak{Z} = z\frac{k_BT}{\lambda^3} \Rightarrow \frac{P}{k_BT} = \frac{z}{\lambda^3}
  \end{equation*}
  and:
  \begin{equation*}
    \langle N\rangle = z \frac{\partial}{\partial z} \left[ \frac{zV}{\lambda^3} \right] = \frac{zV}{\lambda^3} \Rightarrow \frac{\langle N \rangle}{V} = \frac{z}{\lambda^3}
  \end{equation*}
  From which we deduce:
  \begin{equation*}
    \frac{P}{k_BT} = \frac{\langle N \rangle}{V} \Rightarrow PV = \langle N \rangle k_BT
  \end{equation*}
\end{minipage}

3.
\begin{minipage}[t]{0.9\textwidth}
  We have:
  \begin{align}
                & \frac{1}{\lambda^3} \sum_{l=1}^\infty b_l z^l = \frac{P}{k_BT} = \frac{1}{\lambda^3} \sum_{l =1}^\infty a_l \times \left(\lambda^3 \frac{\langle N\rangle}{V} \right)^l \nonumber\\
    \Rightarrow & \sum_{l=1}^\infty b_l z^l =\sum_{l =1}^\infty a_l \times \left(\sum_{k=1}^\infty k b_k z^k\right)^l \label{eq:coeff}
  \end{align}
  Now we match coefficients of $z$. The only way to get a $z$ term on the right handside is if $l = k = 1$:
  \begin{equation*}
    b_1z = a_1 \times b_1 z \Rightarrow a_1 = 1
  \end{equation*}
  Also note that:
  \begin{equation*}
    b_1 = \frac{1}{V}\int_Vd^3r = 1 \Rightarrow a_1=b_1 = 1
  \end{equation*}
  For the $z^2$ terms, the only way to get $z^2$ on the right of Eq.~\ref{eq:coeff} is if $l = 1$ and $k = 2$ or if $l = 2$ and $k = 1$
  \begin{align*}
                & b_2z^2 = a_1 \times (2 b_2 z^2) + a_2 \times (b_1 z^1)^2\\
    \Rightarrow & b_2 = 2 b_2 + a_2 \\
    \Rightarrow & a_2 = - b_2
  \end{align*}
  Where I used the fact that $a_1 = b_1 = 1$.
  For the $z^3$ terms, the only way to get $z^3$ on the right of Eq.~\ref{eq:coeff} is if $l = l$ and $k = 3$ or if $l = 2$ and $k = 1, 2$ (cross multiplied terms) or if $l = 3$ and $k = 1$. 
  \begin{align*}
                & b_3z^3 = a_1 \times (3 b_3 z^3) + a_2 \times (b_1 z \times 2 b_2 z^2 + 2 b_2 z^2 \times b_1 z) + a_3 \times (b_1 z)^3\\
    \Rightarrow & b_3 = 3 b_3 + (-b_2)(4 b_2) + a_3\\
    \Rightarrow & a_3 = 4 b_2^2 - 2 b_3
  \end{align*}
\end{minipage}

4.
\begin{minipage}[t]{0.9\textwidth}
  We wish to find $a_2 = -b_2$, to this end note that:
  \begin{align*}
    b_2 &= \frac{\lambda^3}{2!V\lambda^{3 \times 2}} \int d^3q_1\int d^3q_2f_{12}\\
        &= \frac{1}{2V\lambda^{3}} \int d^3q_1 \int d^3(q_2 - q_1)\left[e^{-\beta V(|q_2 - q_1|)} - 1\right]\\
        &= \frac{1}{2V\lambda^{3}} \int d^3q_1 \int d^3x \left[e^{-\beta V(|x|)} - 1\right]\\
  \end{align*}
  Where $x = q_2 - q_1$.
  Provided that the interaction range of the potential is short the integrals can be treated as approximately independent.
  This follows from the fact that if $q_1$ starts differing largely from $q_2$, the integrand dependent on $x$ becomes negligible ($V \to 0$). Thus the integral over $x$ cannot vary much with changing $q_1$. Hence:
  \begin{align*}
    b_2 &\approx \frac{1}{2V\lambda^{3}} \left(\int d^3q_1\right)\left( \int d^3x \left[e^{-\beta V(|x|)} - 1\right]\right)\\
        &= \frac{1}{2\lambda^{3}} \int d^3x \left[e^{-\beta V(|x|)} - 1\right]\\
        &= \frac{1}{2\lambda^{3}} \int_0^{2\pi} d \phi \int_0^\pi \sin \theta d\theta\int_0^\infty r^2dr \left[e^{-\beta V(|r|)} - 1\right]\\
        &= \frac{2\pi}{\lambda^{3}} \int_0^\infty r^2dr \left[e^{-\beta V(|r|)} - 1\right]
  \end{align*}
  Now we can use this to evaluate $a_2$. 
  \begin{align*}
    a_2  = -b_2 
        &= -\frac{2 \pi}{\lambda^3} \int_0^\infty(e^{-\beta V(r)} - 1)r^2dr\\
        &= -\frac{2 \pi}{\lambda^3} \int_0^\alpha(0 - 1)r^2dr -\frac{2 \pi}{\lambda^3}\int_\alpha^\infty(1 - 1)r^2dr\\
        &= \frac{2 \pi}{\lambda^3} \int_0^\alpha r^2dr\\
        &= \frac{2 \pi \alpha^3}{3\lambda^3}\\
  \end{align*}
\end{minipage}

5.
\begin{minipage}[t]{0.9\textwidth}
  Doing a similar calculation to the above:
  \begin{align*}
    a_2 &= -\frac{2 \pi}{\lambda^3} \int_0^\infty(e^{-\beta V(r)} - 1)r^2dr\\
        &= -\frac{2 \pi}{\lambda^3} \left[\int_0^\alpha(0 - 1)r^2dr + \int_\alpha^{2\alpha}(e^{\beta v_0/kT} - 1)r^2dr - \int_{2\alpha}^\infty(1 - 1)r^2dr\right]\\
        &= -\frac{2 \pi}{\lambda^3} \left[-\frac{\alpha^3}{3} + (e^{\beta v_0} - 1)\left(\frac{8 \alpha^3}{3} - \frac{\alpha^3}{3}\right)\right]\\
        &= -\frac{2 \pi \alpha^3 (7e^{v_0/kT} - 8 )}{3\lambda^3}
  \end{align*}
  It can been seen from this that $a_2$ is negative for low $T$, and is positive for high $T$ when the exponential is large and small respevtively.
  Thus $a_2$ does change sign with temperature.
\end{minipage}

\end{document}
