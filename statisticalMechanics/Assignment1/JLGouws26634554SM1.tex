\documentclass[a4paper,12pt]{article}

\usepackage{amsmath,amsfonts,amsthm,amssymb, mathtools,steinmetz, gensymb, siunitx}	% LOADS USEFUL MATH STUFF
\usepackage{xcolor,graphicx}
\usepackage[a4paper]{geometry} 				% ADJUSTS PAGE
\usepackage{setspace}
\usepackage{physics}
\usepackage{caption}
\usepackage{tikz}
\usepackage{pgf,tikz,pgfplots}
\usepackage{mathrsfs}
\usepackage{amsbsy}
\usepackage{fancyhdr}
\usepackage{float}
\usepackage{array}
\usepackage{booktabs}
\usepackage{newpxtext}
\usepackage{unicode-math}
\usepackage{braket}
\usepackage{tensor}
\usepackage{tabstackengine}
\setmathfont{Libertinus Math}

\usetikzlibrary{decorations.pathreplacing,decorations.markings}
\usepgfplotslibrary{fillbetween}

\newgeometry{left=1cm,top = 2.5cm, bottom = 1.75cm, right = 1cm}

\newcommand{\defeq}{:=}
\newcommand\block[1]{\hspace*{#1}}
\newcommand{\rpm}{\sbox0{$1$}\sbox2{$\scriptstyle\pm$}
	  \raise\dimexpr(\ht0-\ht2)/2\relax\box2 }
\newcommand{\af}{\pmb{\hat a_1}}
\newcommand{\as}{\pmb{\hat a_2}}
\newcommand{\at}{\pmb{\hat a_3}}
\newcommand\uv[1]{\pmb{\hat {#1}}}
\newcommand\vect[1]{\pmb{{#1}}}
\newcommand\dprod{\pmb{\cdot}}

\setstackEOL{;}% row separator
\setstackTAB{,}% column separator
\setstacktabbedgap{1ex}% inter-column gap
\setstackgap{L}{1.0\normalbaselineskip}% inter-row baselineskip
\newcommand\rbm[1]{\left(\Matrixstack{#1}\right)}
	  
\pgfplotsset{compat=newest}
\newlength{\QNo}
\settowidth{\QNo}{2.}

\newlength{\QLetter}
\settowidth{\QLetter}{(a)}

\pagestyle{fancy}
\rhead{Statistical Physics Problem Set}
\lhead{J. L. Gouws}


\begin{document}
\fontencoding{T1}
\fontfamily{ppl}\selectfont
{\Large \textbf{Statistical Physics Tutorial 1}} \hfill {\Large \textbf{J L Gouws}}\\
\block{1.0cm} {\large \textbf{\today}} \hfill {\large \textbf{26634554}}\\
\thispagestyle{empty}
\fontencoding{T1}

1.
\begin{minipage}[t]{0.9\textwidth}
  We have:
  \begin{align*}
    z \frac{ \partial \ln \mathcal{Z}}{\partial z} = z\left. \frac{\partial \mathcal{Z}}{\partial z} \right/ \mathcal{Z} = \frac{z}{\mathcal{Z}} \sum_{N} N z^{N - 1} Z_N = \sum_{N} N \frac{z^{N} Z_N}{\mathcal{Z}}= \sum_{N} N \frac{e^{\beta \mu N} \sum_re^{-\beta E_r}}{\mathcal{Z}} = \sum_{N} N P_N = \langle N \rangle
  \end{align*}
\end{minipage}

2.
\begin{minipage}[t]{0.9\textwidth}
  The partition function for the ideal gas:
  
\end{minipage}

$\phantom{\text{2.}}$
\begin{minipage}[t]{0.9\textwidth}
  b).
  \begin{minipage}[t]{\textwidth}
    $\trace\left({A^\dagger B}\right)$ is an inner product:
    First, $\left(A, B\right) = \trace\left(A^\dagger B\right)$, is indeed a map $\mathbb{C}^{2\times 2}\times \mathbb{C}^{2\times 2} \to \mathbb{C}$.\\
    Let:
    \begin{equation*}
      A = \rbm{a_{11}, a_{12}; a_{21}, a_{22}} \qquad \qquad B = \rbm{b_{11}, b_{12}; b_{21}, b_{22}} \qquad  \qquad C = \rbm{c_{11}, c_{12}; c_{21}, c_{22}}
    \end{equation*}
    Note:
    \begin{align*}
      \left( A, B \right) &= \trace\left\{\rbm{a_{11}^*, a_{21}^*; a_{12}^*, a_{22}^*} \rbm{b_{11}, b_{12}; b_{21}, b_{22}}\right\}\\
      & = \trace\left\{\rbm{a_{11}^*b_{11} + a_{21}^* b_{21}, a_{11}^*b_{12} + a_{21}^*b_{22}; a_{12}^*b_{11} + a_{22}^*b_{21}, a_{12}^*b_{12} + a_{22}^* b_{22}} \right\} \\
      &= a_{11}^*b_{11} + a_{21}^* b_{21} + a_{12}^*b_{12} + a_{22}^* b_{22}
    \end{align*}
    Similarly
    \begin{align*}
      \left(B, A \right) = b_{11}^*a_{11} + b_{21}^* a_{21} + b_{12}^*a_{12} + b_{22}^* a_{22}
    \end{align*}
    It is now clear that:
    \begin{equation*}
      \left( B, A \right)^* = \left(b_{11}^*a_{11} + b_{21}^* a_{21} + b_{12}^*a_{12} + b_{22}^* a_{22}\right)^* = a_{11}^*b_{11} + a_{21}^* b_{21} + a_{12}^*b_{12} + a_{22}^* b_{22} = \left( A, B\right)
    \end{equation*}
    As required, now moving onto linearity in the second argument:
    \begin{align*}
      \left(A , cB + dC \right) &= \trace\left\{A^\dagger\left(cB + dC\right)\right\}\\
                    &= \trace\left\{\rbm{a_{11}^*, a_{21}^*; a_{12}^*, a_{22}^*} \rbm{cb_{11} + dc_{11}, cb_{12} + d c_{12}; cb_{21}+ d c_{21}, cb_{22}+ d c_{22}}\right\}\\
                    &= a_{11}^*(cb_{11} + dc_{11}) + a_{21}^* (cb_{21} + dc_{21}) + a_{12}^*(cb_{12} + dc_{12}) + a_{22}^* (cb_{22}+ dc_{22})\\
                    &= c\left(a_{11}^*b_{11} + a_{21}^* b_{21} + a_{12}^*b_{12} + a_{22}^*b_{22} \right)+ d\left(a_{11}^*c_{11}  + a_{21}^* c_{21}  + a_{12}^*c_{12} + a_{22}^*c_{22}\right)\\
                    &= c\trace(A^\dagger B) + d\trace(A^\dagger C)\\
                    &= c\left(A, B\right) + d\left(A, C\right)
    \end{align*}
    Positive definiteness is shown by:
    \begin{align*}
      \left(A , A \right) &= a_{11}^*a_{11} + a_{12}^*a_{12} + a_{21}^*a_{21} + a_{22}^*a_{22}\\
                          &= |a_{11}|^2 + |a_{12}|^2 + |a_{21}|^2 + |a_{22}|^2\\
                          & \geq 0
    \end{align*}
    and since $|c|^2 = 0 \Leftrightarrow c = 0$ for any $c \in \mathbb{C}$.
    We have form the above:
    \begin{align*}
      \left(A , A \right) = 0 \Leftrightarrow A = \rbm{0, 0; 0,0}
    \end{align*}
    This is actually obvious if one realizes that there is an isomorphism:
    \begin{align*}
      I :& \mathbb{C}^{2 \times 2} \to \mathbb{C}^4\\
        :& \rbm{a, b; c, d} \mapsto \rbm{a; b; c; d}
    \end{align*}
    Then $\tr{A^\dagger B}$ is the standard inner product on $\mathbb{C}^4$.
  \end{minipage}
\end{minipage}

\begin{minipage}[t]{0.9\textwidth}
  $\phantom{\text{b).}}$
  \begin{minipage}[t]{\textwidth}
    The product of a $2\times 2$ diagonal matrix and an anti-diagonal matrix zero:
    \begin{equation*}
      \rbm{a, 0; 0, b} \rbm{0, c; d, 0} = \rbm{0, 0; 0, 0} \Rightarrow (I, \sigma_x) = (I, \sigma_y) = (\sigma_z, \sigma_x) = (\sigma_z, \sigma_y) = 0
    \end{equation*}
    It now remains to check the product of the two diagonal and anti-diagonal matrices:
    \begin{gather*}
      (I, \sigma_z) = 1 + 0 + 0 - 1 = 0\\
      (\sigma_x, \sigma_y) = 0 + 1^* (-i) + 1^*(i) + 0 = 0\\
    \end{gather*}
    This shows that the four matrices are mutually orthogonal.\\

    All matrices have two elements, and each element has modulus of $1$.
    This makes the norm of each matrix $\sqrt{2}$, hence:
    \begin{equation*}
      \left\{\frac{I}{\sqrt{2}}, \frac{\sigma_x}{\sqrt{2}} , \frac{\sigma_y}{\sqrt{2}} , \frac{\sigma_z}{\sqrt{2}} \right\}
    \end{equation*}
    Forms an orthonormal basis for this space.\\
  \end{minipage}
\end{minipage}

3.
\begin{minipage}[t]{0.9\textwidth}
  \begin{minipage}[t]{\textwidth}
    Let the basis of our ket space be given by $\left\{\ket{a}\right\}$, that is the set of eigenkets of $A$.
    Now:
    \begin{equation*}
      A^n = A^n\sum_a \ket{a}\bra{a} = \sum_a A^n\ket{a}\bra{a} = \sum_a a^n\ket{a}\bra{a}
    \end{equation*}
    And similarly:
    \begin{equation*}
      A^m = \sum_a a^m\ket{a}\bra{a}
    \end{equation*}
    Thus
    \begin{align*}
      A^nA^m &= \left(\sum_a a^n\ket{a}\bra{a}\right)\left(\sum_{a'} a'^m\ket{a'}\bra{a'}\right)\\
             &= \sum_a \sum_{a'} a^na'^m\ket{a}\braket{a | a'}\bra{a'}\\
             &= \sum_a \sum_{a'} a^na'^m\ket{a}\tensor{\delta}{^{a'}_{a}}\bra{a'}\\
             &= \sum_a a^na^m\ket{a}\bra{a}\\
             &= \sum_a a^{n + m}\ket{a}\bra{a}\\
             &\equiv A^{n + m}
    \end{align*}
  \end{minipage}
\end{minipage}

4.
\begin{minipage}[t]{0.9\textwidth}
  a).
  \begin{minipage}[t]{\textwidth}
    By definition:
    \begin{equation*}
      \trace{A} = \sum_i \braket{i|A|i}
    \end{equation*}
    Now:
    \begin{align*}
      \trace(AB) &= \sum_i \braket{i|AB|i}\\
                 &= \sum_i\sum_j \braket{i|A|j}\braket{j|B|i}\\
                 &= \sum_i\sum_j \braket{j|B|i}\braket{i|A|j}\\
                 &= \sum_j \braket{j|BA|j}\\
                 &= \trace(BA)
    \end{align*}
  \end{minipage}

  b).
  \begin{minipage}[t]{\textwidth}
    We have:
    \begin{align*}
      AB &= \sum_{i} AB\ket{i}\bra{i}\\
         &= \sum_{i, j}A\ket{j} \bra{j}B\ket{i}\bra{i}\\
         &= \sum_{i, j, k}\ket{k} \bra{k}A\ket{j} \bra{j}B\ket{i}\bra{i}\\
         &= \sum_{i, j, k}\left( \bra{k}A\ket{j} \bra{j}B\ket{i} \right) \ket{k}\bra{i}
    \end{align*}
    Now last expression above is just the sum of a number, $\braket{k|A|j}\braket{j|B|i}$, times by an operator and so:
    \begin{align*}
      (AB)^\dagger &= \sum_{i, j, k} \left[\left(\bra{k}A\ket{j} \bra{j}B\ket{i} \right)\ket{k}\bra{i}\right]^\dagger\\
                   &= \sum_{i, j, k} \left(\bra{k}A\ket{j} \bra{j}B\ket{i}\right)^* \ket{i}\bra{k}\\
                   &= \sum_{i, j, k}\left( \bra{j}A^\dagger\ket{k} \bra{i}B^\dagger\ket{j}\right) \ket{i}\bra{k}\\
                   &= \sum_{i, j, k} \ket{i}\bra{i}B^\dagger\ket{j}\bra{j}A^\dagger\ket{k}\bra{k}\\
                   &= \sum_{i}\ket{i}\bra{i}B^\dagger A^\dagger\\
                   &= B^\dagger A^\dagger
    \end{align*}
    As required.
  \end{minipage}

  c).
  \begin{minipage}[t]{\textwidth}
    \begin{align*}
      A &= \ket{\alpha}\bra{\beta}\\
        &= \sum_n\ket{n}\braket{n|\alpha}\bra{\beta}\\
        &= \sum_n\sum_m\ket{n}\braket{n|\alpha}\braket{\beta|m}\bra{m}
    \end{align*}
    This is a matrix with the $nm$-th element having the value: $\braket{n|\alpha}\braket{\beta|m}$.
  \end{minipage}
\end{minipage}

$\phantom{\text{4.}}$
\begin{minipage}[t]{0.9\textwidth}
  d).
  \begin{minipage}[t]{\textwidth}
    \begin{align*}
      A^\dagger &= \left(\sum_n\sum_m\ket{n}\braket{n|\alpha}\braket{\beta|m}\bra{m}\right)^\dagger\\
                &= \sum_n\sum_m \left(\braket{n|\alpha}\braket{\beta|m}\right)^*\ket{m}\bra{n}\\
                &= \sum_n\sum_m \left(\braket{m|\alpha}\braket{\beta|n}\right)^*\ket{n}\bra{m}\\
    \end{align*}
    This shows that the $nm$-th element of $A^\dagger$ is the complex conjugate of the $mn$-th element of $A$. This is the definition of the Hermitian conjugate of a matrix.
  \end{minipage}
\end{minipage}
\end{document}
